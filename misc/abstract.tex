\thispagestyle{empty}
\vspace*{1.0cm}

\begin{center}
    \textbf{Abstract}
\end{center}

\vspace*{0.5cm}

\noindent
Blockchains faced an increase of usage in recent years, posing new challenges to the scalability and efficiency of the system. Several congestion incidents have happened in the past, making blockchains slow and incredibly costly. To mitigate these episodes some internal modifications of the blockchains have been proposed, known as Layer 1 solutions. After being revealed quite inefficient some other type of enhancements that exploit resources outside the blockchain have been studied, called Layer 2 solutions. Between those Layer 2 solutions there are rollups: they consist in executing a bunch of transactions outside the blockchain, and sending just the result of the rollup computation to the blockchain. This approach allows just one blockchain interaction per group of transaction instead of one per every transaction. This allows to save on fees and reduce the congestion on the blockchain.

One type of Rollups are Zero-Knowledge Rollups (Zk-Rollups): they allow executing a rollup and send a proof of correct execution to the blockchain, that will verify it and include the rollup results in its state. One problem with this type of rollup is they are very resource demanding for the Layer 2 system. To ease the creation of Zero-Knowledge programs some toolbox are available: one of them is ZoKrates, developed at the Technical University of Berlin.

This master thesis proposes a Zk-Rollup system written using the ZoKrates toolbox, that can be applied to the vast majority of blockchains with little modifications. The Rollup System is deployed for the testing and development on the Tezos Blockchain. The system shows its capability on handling different types of tokens, like in the case of an NFT marketplace and on saving on fees, while remaining usable during congestions. It resulted also to be resource-efficient, allowing to run it on a wide range of hardware.


% This template is intended to give an introduction of how to write diploma and master thesis at the chair 'Architektur der Vermittlungsknoten' of the Technische Universität Berlin. Please don't use the term 'Technical University' in your thesis because this is a proper name. 
% \\
% \\
% On the one hand this PDF should give a guidance to people who will soon start to write their thesis. The overall structure is explained by examples. On the other hand this text is provided as a collection of LaTeX files that can be used as a template for a new thesis. Feel free to edit the design.
% \\
% \\
% It is highly recommended to write your thesis with LaTeX. I prefer to use Miktex in combination with TeXnicCenter (both freeware) but you can use any other LaTeX software as well. For managing the references I use the open-source tool jabref. For diagrams and graphs I tend to use MS Visio with PDF plugin. Images look much better when saved as vector images. For logos and 'external' images use JPG or PNG. In your thesis you should try to explain as much as possible with the help of images.
% \\
% \\
% The abstract is the most important part of your thesis. Take your time to write it as good as possible. Abstract should have no more than one page. It is normal to rewrite the abstract again and again, so  probaly you won't write the final abstract before the last week of due-date. Before submitting your thesis you should give at least the abstract, the introduction and the conclusion to a native english speaker. It is likely that almost no one will read your thesis as a whole but most people will read the abstract, the introduction and the conclusion.
% \\
% \\
% Start with some introductionary lines, followed by some words why your topic is relevant and why your solution is needed concluding with 'what I have done'. Don't use too many buzzwords. The abstract may also be read by people who are not familiar with your topic.