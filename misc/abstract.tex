\thispagestyle{empty}
\vspace*{1.0cm}

\begin{center}
    \textbf{Abstract}
\end{center}

\vspace*{0.5cm}

\noindent
Blockchains faced an increase of usage in recent years, posing new challenges to the scalability and efficiency of the system. Several congestion incidents have happened in the past, making blockchains slow and incredibly costly. To mitigate these episodes some internal modifications of the blockchains have been proposed, known as Layer 1 solutions. After being revealed quite inefficient some other type of enhancements that exploit resources outside the blockchain have been studied, called Layer 2 solutions. Among Layer 2 solutions, there are rollups. These involve executing groups of transactions off the blockchain and only sending the results of the rollup computation to the blockchain. This approach reduces the number of interactions with the blockchain, which means lower fees and less congestion.

One specific type of Rollup is called Zero-Knowledge Rollups (Zk-Rollups). They allow off-chain execution of transactions, followed by proof submission to the blockchain for verification and inclusion in the rollup state. However, they can be very resource-intensive for Layer 2 systems. To ease the creation of Zero-Knowledge programs, various tools are available: one of them is ZoKrates, developed at the Technical University of Berlin.

This master thesis proposes a Zk-Rollup system built using the ZoKrates toolbox. It's designed to be easily adaptable to the vast majority of blockchains with minimal changes. The Rollup System is deployed for testing and development on the Tezos Blockchain. The system demonstrates its ability to handle different types of tokens, such as in an NFT marketplace, while reducing transaction fees and maintaining usability during peak periods. It also proved to be resource-efficient, making it usable on a wide range of hardware, and more cost-effective than standard blockchain transactions.

