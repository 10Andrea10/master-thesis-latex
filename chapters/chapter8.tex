\chapter{Conclusion\label{cha:chapter7}}
% The final chapter summarizes the thesis. The first subsection outlines the main ideas behind Component X and recapitulates the work steps. Issues that remained unsolved are then described. Finally the potential of the proposed solution and future work is surveyed in an outlook.
% Explain what you did during the last 6 month on 1 or 2 pages!

\section{Summary\label{sec:summary}}

The work done in this master thesis allowed to show how a problem such as high transactions fee in the blockchain can be solved, by introducing a new layer on top of the blockchain, called Layer 2. The proposed solution presents a simple architecture adaptable to various blockchains with minimal adjustments to Layer 1. In particular, the system can accommodate diverse token types, including non-fungible tokens (NFTs), like in the case of an NFT marketplace. These achievements align with the project's requirements outlined in Chapter \ref{cha:chapter3}, where is given remarkable importance to the need of having an easily adaptable system.

\noindent The work can be summarized into the following work steps:
\vspace{-0.11in}
\begin{itemize}
	\item Analysis of existing Zk-Rollup projects and their solutions to common problems;
	      \vspace{-0.11in}
	\item Examination of functionalities and limitations of the Tezos blockchain;
	      \vspace{-0.11in}
	\item Study of ZoKrates toolbox and programs;
	      \vspace{-0.11in}
	\item Design and implementation of Layer 2 functionalities;
	      \vspace{-0.11in}
	\item Implementation of enhancements to Layer 2;
	      \vspace{-0.11in}
	\item Development of an efficient Layer 1 to support the new Layer 2;
	      \vspace{-0.11in}
	\item Benchmarking of single components and the entire system;
	      \vspace{-0.11in}
	\item Evaluation of the Rollup System.
\end{itemize}

The final Rollup System is capable of handling a considerable user load, tested up to 2024, and a proportional volume of transactions, tested up to 55 transactions per rollup. Results indicate a 40\% reduction in transaction fees per transaction compared to normal Layer 1 transactions, with costs decreasing linearly as transactions are included within the rollup.

Witness and proof generation times heavily depend on the underlying CPU hardware running ZoKrates. Test results showed a maximum of 40 minutes for generating a witness and proof for 1024 users and 30 transactions. Reduction of these extended execution times can be achieved through more powerful CPUs, leveraging cloud services for short bursts of computation. Notably, when compared to the alternative Optimistic Rollup solution, our Zk-Rollup system boasts significantly lower confirmation periods.

\section{Challenges Faced}

Several challenges were encountered during the project's progression, preventing its full functionality. The most important among these challenges are:
\vspace{-0.11in}
\paragraph{Hash Function} The choice of hash function revealed to be critical in hardware resource reduction. Transitioning from sha2-256 to Poseidon hash allowed a remarkable 10000\% decrease in RAM usage. This achievement allowed to run the system for a wider number of users, from an initial 8 users to up to 2024 users.
\vspace{-0.11in}
\paragraph{Proof Reduction and Compression} As user numbers grew, so did proof sizes, preventing its upload to the verifier smart contract. To solve this issue, a proof reduction and compression algorithm was implemented, reducing proof length from \textit{\#users*2} to \textit{\#transactions}. Although this approach works from Layer 2 side and verification phase, it is not yet possible to extract the compressed data to update the L1 storage due to missing functionalities in the SmartPy library.
\vspace{-0.11in}
\paragraph{Repeated Transition Attack} Malicious users could send repeated transactions, draining sender's account even with valid signatures. To counteract this, a nonce mechanism was introduced, preventing the same transaction to be executed more than once through nonce increment, allowing the system to check if the transaction has already been executed.

\section{Broad Usage\label{sec:dissemination}}

This project successfully proved the feasibility of implementing a Tezos-based Zk-Rollup system using the ZoKrates toolbox. The system's cost efficiency is superior than conventional L1 transactions, with the possibility to apply the system to different token types, like in the case of an NFT marketplaces. This versatility allows projects with different scopes to adopt the system in various blockchains and applications, requiring modest hardware capabilities, proportional of the amount of users and transactions foreseen in the application.

\section{Future Work}

Future work should prioritize the following areas:
\vspace{-0.11in}
\begin{itemize}
	\item \textbf{Proof Reduction and Compression}: Complete the proof reduction and compression algorithm for Layer 1, allowing contract storage updates with new balances and tokens;
	      \vspace{-0.11in}
	\item \textbf{Rollup System Optimization}: Optimize the Rollup System to reduce witness and proof generation times. This can be achieved by ZoKrates program optimization and performance evaluations on different CPU types and architectures;
	      \vspace{-0.11in}
	\item \textbf{Rollup System Cost}: Study infrastructure operation costs to evaluate a fair transaction fee to reward nodes submitting valid proofs.
	      \vspace{-0.11in}
\end{itemize}




% Future work will enhance Component X with new services and features that can be used ...