\chapter{Introduction}\label{cha:chapter1}

This chapter provides an overview of the motivation, context, and objectives of the thesis.

\section{Motivation}\label{sec:moti}

In recent years, blockchain technology has witnessed remarkable growth, with the introduction of Bitcoin in 2008 \cite{nakamoto_bitcoin_2008} marking the emergence of a decentralized, distributed ledger concept. Since then, numerous blockchains with unique features have been developed, gaining widespread popularity.

As the popularity of blockchains continues to rise, many companies are exploring the technology's potential and attracting new users. Predictions indicate exponential growth in the blockchain space in the coming years \cite{noauthor_global_nodate}. However, this growth presents a challenge due to the limited scalability offered by current blockchains. For instance, Ethereum, one of the most popular blockchains, currently faces a transaction processing limit of only 15 transactions per second (TPS).

The necessity for blockchain scalability has been recognized by the community, especially evident when Ethereum transaction fees reached an average of \$70 in 2020\footnote{\url{https://ycharts.com/indicators/ethereum_average_transaction_fee}} due to the high volume of submitted transactions and consequent congestion of the network. This issue is not unique to Ethereum, as other blockchains also experience similar problems. Vitalik Buterin's `Scalability Trilemma' assert that achieving scalability, security, and decentralization simultaneously in a distributed system, such as blockchains, is a challenging task\footnote{\url{https://ethereum.org/en/roadmap/vision/}}.

\section{Objective}\label{sec:objective}

The objective of this thesis is to address the issue of limited scalability and high fees in blockchains. A crucial factor that constrains the number of transactions in a block is gas limit. The gas limit serves as a hard limit on the computational work that can be performed within a block, measured in gas units. Each transaction and contract interaction, in fact, consumes gas, so gas limit determines the maximum amount of computational availability.

The gas limit mechanism is essential for preventing infinite loops in a smart contract execution and defending against denial-of-service attacks. It's also necessary to ensure that a minimal number of transactions can be included in the block. However, it can also inhibit the proper execution of a program that intentionally involves extended computations \cite{wood_ethereum_nodate}.

Enhancing the blockchain's throughput is also directly impacted by the block size, as increasing its size can increase the system's capacity. Nonetheless, a thorough analysis of the consequences of enlarging the block size is necessary, as arbitrary increases can lead to negative effects. For instance, large block sizes may cause network congestion, slower propagation speeds, and an increase in the number of stale blocks—blocks that are not included in the longest chain due to conflicts or concurrency—posing potential security risks \cite{gervais_security_2016}.

This thesis proposes a solution to the scalability problem by introducing a new block-chain Layer that lives on top of the original blockchain. This is called a Layer 2 solution (L2). This second Layer is entirely run by nodes that are not involved in computing or verifying Layer 1 transactions. This way we are not constrained by limitations imposed by the chain, like the gas limit or block size, but we are free to use a lot more computational power. Rollups, a Layer 2 solution, in fact inject into the blockchain only the result of the computations performed outside the chain, making use of the unconstrained computational availability offered on L2. With this approach it should be possible to achieve higher throughput when congestion occurs and lower transaction fees.

This thesis poses three key milestones for achieving the objective of obtaining a functional and efficient Rollup System. These primary objectives can be summarized as follows:
\begin{itemize}
    \vspace{-0.11in}
    \item finalize the zk-rollup project upon which this system is built;
          \vspace{-0.11in}
    \item conduct an evaluation, enhancement, and benchmark of the rollup system.
          \vspace{-0.11in}
    \item create and build a prototype customized to run a NFT marketplace.
\end{itemize}\

A deeper explanation about problem statements and research questions is given in Chapter \ref{sec:4_problemrefinement}.

\section{Structure\label{sec:structure}}

\todo[inline]{Change}

This section gives a brief introduction into the main chapters of this thesis.
\\
\\
\textbf{Chapter \ref{cha:chapter2}} analyzes the current state of the art in the field of blockchain scalability and presents the most relevant related work including the project this thesis is based on.
\\
\\
\textbf{Chapter \ref{cha:chapter3}} analyzes the research question and the requirements for the Rollup System.
\\
\\
\textbf{Chapter \ref{cha:chapter4}} describes the architecture of the Layer 1 and Layer 2 components. It also describes the communication between the two layers.
\\
\\
\textbf{Chapter \ref{cha:chapter5}} describes the implementation part of the Rollup System.It shows the components and the technologies used to build the system. It also describes the challenges faced during the development.
\\
\\
\textbf{Chapter \ref{cha:chapter6}} shows the tests performed and the results obtained. It also shows the limitations of the system and the possible future improvements.
\\
\\
\textbf{Chapter \ref{cha:chapter7}} summarizes the work done and the challenges faced. It presents the conclusions and the future work to improve the system.
