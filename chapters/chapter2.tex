\chapter{State of the Art \& Related Work}
\label{cha:chapter2}


This chapter gives a look at the main ideas in blockchain scalability and introduces the core project on which this thesis is based.

\section{Scalability Approaches}
Various strategies are being explored to enhance blockchain scalability. Some are already operational, while others are in the research stage. These strategies broadly fall into two categories: Layer 1 and Layer 2 solutions \cite{tyagi_study_2021,thibault_blockchain_2022}. Layer 1 solutions focus on boosting blockchain scalability through means like enlarging block sizes or implementing Sharding. In contrast, Layer 2 solutions attempt to scale the blockchain by relocating computation off-chain. Noteworthy examples of Layer 2 solutions are State Channels, Sidechains, and Rollups, as illustrated in Figure \ref{fig:2_scalingSolutions}.

The common path for making blockchains handle more transactions is to use Layer 2 solutions. Many members of the Ethereum community are following this approach \cite{neiheiser_practical_2023}.

\begin{figure}[ht]
  \centering
  \includegraphics[width=0.95\columnwidth]{2_scalingSolutions.jpg}
  \caption[Scaling Solutions]{Principal scaling solutions for Layer 1 and Layer 2\footnotemark}  
  \label{fig:2_scalingSolutions}
\end{figure} 
\footnotetext{\url{https://medium.com/token-terminal/a-primer-on-ethereum-l2-scaling-techniques-17ac437891b1}}

\subsection{Zk-Rollups}
\label{sec:2_zkRollups}
Zk-Rollups represent a Layer 2 approach that leverages zero-knowledge proofs to execute computation off-chain. Post-computation outcomes and a proof of verifiably correct execution are transmitted to the blockchain's smart contract. The smart contract then verifies the proof before incorporating the results onto the blockchain \cite{tyagi_study_2021}.

Figure \ref{fig:2_zkRollup_schema} presents the Zk-Rollup paradigm: 
\begin{enumerate}
  \item Users initiate transactions with the off-chain node;
  \item The node processes the transaction batch, using the state fetched from the blockchain;
  \item The node forwards results and proof to the smart contract;
  \item The smart contract executes a proof validation;
  \item If successful, the smart contract updates the state.
\end{enumerate}

\begin{figure}[ht]
  \centering
  \includegraphics[width=0.95\columnwidth]{2_zkRollup_architecture.png}
  \caption[Zk-Rollup Schema]{General schema of a Zk-Rollup\cite{ise_department_tub_material_nodate}}  
  \label{fig:2_zkRollup_schema}
\end{figure}

\subsection{ZoKrates}
ZoKrates is a toolbox for zkSNARKs on Ethereum: it allows developers to write zero-knowledge proofs in a high-level programming language and generate trusted setup parameters and zero-knowledge proofs \cite{eberhardt_ZoKrates_2018}. It is in fact necessary to rely on a trusted setup phase to generate the proving and verifying keys, which are used to generate and verify the proofs.

This toolbox utilizes a C++ language style, facilitating developers adoption. It also offers a built-in library with frequently used functions like hash functions, elliptic curve operations, and Merkle tree operations. The toolbox is also capable of generating a Solidity contract from the ZoKrates program, deployable on the Ethereum blockchain. The contract can then be used to verify the proofs generated by the ZoKrates program. However, a drawback of this toolbox is the limited supported data types: due to the resource-intensive phases like computing the witness and generating a proof, the program must avoid using complex and memory-consuming types. For this reason the primitive types supported are integers, booleans and field values, whose size depends on the underlying elliptic curve.

Another constraint is that the toolbox requires loop sizes to be known during compile time. This becomes problematic when dealing, for instance, with a variable number of users, as the number of iterations of the loops depends on the number of users.

\section{ADSP Project: Scaling Tezos Blockchain with Zk-Rollups \label{sec:2_adspProject}}

In the Winter Semester 2022/2023 at the University of Berlin, I collaborated with the ADSP (Advanced Distributed Systems Prototype) group on a project. The project's objective was to scale the Tezos blockchain using Zk-Rollups through the ZoKrates toolbox.

The group partially achieved the project's objectives, creating a proof of concept. The system architecture features Layer 1 and Layer 2 components: Layer 1 includes two smart contracts for state management and proof verification; Layer 2 encompasses a node executing transactions within the ZoKrates program, generating proofs for the smart contract. Figure \ref{fig:2_general_rollup_architecture} offers an overview of the entire system.

\begin{figure}[ht]
  \centering
  \includegraphics[width=0.95\columnwidth]{2_drawings-adsp_rollup_architecture.png}
  \caption[Project Architecture]{Project architecture}  
  \label{fig:2_general_rollup_architecture}
\end{figure}

The Zk-Rollup execution process occurs as follows:
\begin{enumerate}
    \item A user dispatches a transaction to the off-chain node, managed by a NodeJS server (the Web-Manager);
    \item The Web-Manager logs the transaction in a transaction queue;
    \item Once three transactions accumulate, the Web-Manager fetches the current state from the Tezos Zk-Rollup smart contract;
    \item The L2 node processes the transaction batch, executing the ZoKrates program using the transaction batch and the current state as input;
    \item The ZoKrates program generates a proof and new balance list;
    \item The node converts ZoKrates outputs into a Tezos-compatible format;
    \item Converted outputs are dispatched to the Tezos Rollup smart contract using Taquito, an external Typescript library;
    \item The smart contract forwards the proof to the verifier smart contract;
    \item The verifier smart contract verifies the proof, communicating the result to the Rollup-Manager contract;
    \item If the proof is verified, the new balance list is stored in the Manager contract's blockchain storage.
\end{enumerate}

Figure \ref{fig:2_drawings-adsp_sequence_general.png} illustrates the sequence of events in the Zk-Rollup execution process.

\begin{figure}[ht]
  \centering
  \includegraphics[width=1\columnwidth]{2_drawings-adsp_sequence_general.png}
  \caption[Sequence Zk-Rollup]{Sequence diagram of the Zk-Rollup execution process.}  
  \label{fig:2_drawings-adsp_sequence_general.png}
\end{figure} 

\subsection{Developed Components}
The team successfully implemented the Rollup-Manager \textit{Verify} entrypoint within the \textit{Rollup contract}, along with the \textit{NodeJS Manager}. Additionally, the storage mechanism was correctly executed.

The ZoKrates program was partially developed, including:
\begin{itemize}
    \item Verification of account inclusion in the rollup system;
    \item Verification of sufficient balance in sender account;
    \item Transaction execution;
    \item Generation of new balance merkle tree root, balance list, and proof.
\end{itemize}

The generation of the Merkle tree root is accomplished through the algorithm described below. This process utilizes the SHA2-256 hash function with automatic zero-padding. The algorithm incorporates pre-calculated values for the Merkle tree depth and the number of leaf nodes.

\noindent\textbf{Algorithm 1: Merkle Tree Root Generation}
\begin{lstlisting}[language=Python]
for i in 0...TREE_DEPTH {
  step_size = 2 ** (i + 1)
  step_number = LEAFS_NUM / step_size
  for j in 0...step_number {
    leftIndex = j * step_size
    rightIndex = leftIndex + step_size / 2
    leafs[leftIndex] = hash(
        leafs[leftIndex],
        leafs[rightIndex]
    )
  }
}
return leafs[0]
\end{lstlisting}

\subsection{Pending Components\label{subsec:pendingcomponents}}
Numerous aspects of the project remain unfinished, including:
\begin{itemize}
  \item Registration entrypoint;
  \item Deregistration entrypoint;
  \item Deposit entrypoint;
  \item Withdrawal entrypoint;
  \item Integration of signature checks within the ZoKrates program;
  \item Addressing the double spending issue;
  \item System optimization.
\end{itemize}
Furthermore, the project lacks a benchmarking study, a cost analysis and a proof of concept for a real-world application.

\todo[inline]{Provide a more comprehensive analysis of the project's state of the art.}

\section{Other Zk-Rollup Implementations}
Numerous Zk-Rollup implementations exist, particularly for Ethereum. The implementation discussed in \cite{dinh_implementation_2023} shares similarities with the ADSP project's approach. However, it lacks considerations for system scaling and handling multiple token types. Another example is Polygon Zero\footnote{\url{https://polygon.technology/blog/introducing-plonky2}}, which facilitates parallel execution of transactions on L2. This system blends STARK and SNARK, differing from the approach outlined in this thesis.

\section{Zk-Rollup Reviews}
Several studies have assessed the viability and performance of the Zk-Rollup system. \cite{capko_state_2022} offers an overview of prevalent zero-knowledge proofs. The ZoKrates Toolbox, utilized by the ADSP project, employs SNARK, which requires a trusted setup phase—a potential drawback compared to Transparent zk-SNARK \cite{zhou_overview_2022}. \cite{neiheiser_practical_2023} highlights the scalability bottleneck L2 solutions may face due to a possible slow L1, constraining them below theoretical L2 throughput. \cite{starkware_fractal_2021} proposes a new Layer 3 to overcome this issue.

\section{Alternative L2 Solutions}
Several Layer 2 strategies exist to scale blockchains. A brief overview follows.

Plasma: Plasma facilitates off-chain transaction processing through child chains or sidechains, connected to the main blockchain for security. Plasma chains expedite transactions but necessitate dispute resolution for potentially fraudulent actions \cite{thibault_blockchain_2022}. OMG Network\footnote{\url{https://docs.omg.network}} exemplifies a network employing Plasma.

State Channels: Off-chain payment channels, like state channels, enable direct user transactions without main blockchain interaction. They suit use cases like micropayments and gaming, affording fast, economical transactions \cite{negka_blockchain_2021}. Perun\footnote{\url{https://perun.network/wp-content/uploads/Perun2.0.pdf}} serves as an instance of state channel implementation.

Optimistic Rollups: This type of rollups assumes transaction validity, necessitating verification only during disputes. It combines swifter transactions with main blockchain arbitration \cite{thibault_blockchain_2022}.
